\documentclass[../Main.tex]{subfiles}
\begin{document}

\begin{spacing}{1}
The high costs and privacy implications of data breaches have been widely publicised, but through what mechanisms are these costs imposed on firms, and to what extent do firms internalise them? Do consumers value their privacy enough to punish firms for leaking their private information? This paper uses a data set of over 750 data breaches and an event study methodology to evaluate changes in revenues, expenses, profit, and stock returns that are caused by data breaches. I find evidence that consumers will indeed punish some firms for leaking their data, however, the response is highly heterogeneous and temporary. This punishment does translate into temporarily reduced profits. Furthermore, consistent with previous findings, a firm's stock returns fall significantly in the wake of a data breach. On the other hand there is no evidence that firms change their behaviour as a response to this punishment by increasing spending on security or advertising. In fact, expenses may fall, and this seems to reflect down-scaling or cutbacks which come as a result of data breaches. There is no evidence that any of the identified negative effects persist. The findings of this paper have implications about the sufficiency of market mechanisms in imposing a socially optimal outcome with regards to data security.
\end{spacing}

\biblio % Needed for referencing to working when compiling individual subfiles - Do not remove
\end{document}