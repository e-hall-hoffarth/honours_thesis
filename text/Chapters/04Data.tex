\documentclass[../Main.tex]{subfiles}
\begin{document}

Broadly speaking the primary data used in this paper are data on data breaches themselves as well as data on outcomes for affected firms. Data on data breaches were obtained from Privacy Rights Clearinghouse (\citeyear{privacyrightsclearinghouse2018}). This database lists more than 8000 data breaches since the year 2005, and provides for each of these information on the public announcement date, number of records leaked, type of breach (hacking, lost device, etc.), as well as a small description of the event which appears to come from news headlines and press releases. This list is likely to be very close to a comprehensive list of all events, at least in the United States --- the area that the proposed study focuses on --- because all 50 states have laws that mandate that firms publicly announce the occurrence of a data breach \citep{ncsl2018}. In other words, the occurrence of a data breach is public information, thus a similar or identical list should be possible to obtain from any other source. 

This paper considers the subset of the Privacy Rights Clearinghouse database for which the data breach occurred in the United States and the affected firm is publicly traded\footnote{Those firms for which financial information is readily available}, for the time period January 01, 2005 to December 31, 2017. This results in a final relevant sample size of 759. The number of records leaked is highly skewed to the right as summarised in Table \ref{recordsdata} and shown in Figure \ref{recordsleakedfig}, therefore, the natural logarithm of this variable is always used in regression specifications. It is also noteworthy that a large number of data breaches (358) actually result in zero records being exposed. Data on the type of information leaked is summarised in Table \ref{typesofdataloss}. These variables were created by scraping the event description for the given key words. Therefore, this variable is not perfect and contains a considerable amount of noise, however, it does allow for some novel comparisons.

The primary dependent variables in the \textit{firm study} are firm revenue, profit and expensese\footnote{COMPUSTAT also provides a variable called net quarterly sales, but using this instead of revenue results in nearly identical results.}. These variables were collected for all publicly traded companies listed in the Privacy Rights Clearinghouse database from the quarterly financial statements as reported by COMPUSTAT (\citeyear{compustat}) for the duration of the sample period. These data are summarised at the top of Table \ref{compdata}. In addition to the primary outcomes, COMPUSTAT also provides a number of general firm specific variables that may be relevant such as number of employees and shareholders' equity. Another supply side variable that is potentially relevant is prices, as firms may lower prices to incentivise consumers to come back after the data breach. However, these data proved difficult to obtain and are not essential to include as lower demand will still be reflected in lower revenues and profits, regardless if it is quantity sold or price that actually falls. For the \textit{stock market study}, comprehensive daily stock return data were also obtained through CRSP \citeyear{crsp}, and data for the Fama-French factors were obtained from the personal web-page of Kenneth French (\citeyear{famafrench}). These data are summarised in Table \ref{stockdata}.

Further data on the monthly amount of Google Searches of a firm were collected using an API for Google Trends (\citeyear{googletrends}). In particular, data were collected on an index of searches for the firm's name\footnote{After discarding various uninformative strings such as 'INC'} and its stock ticker. These data are summarised at the bottom of Table \ref{compdata}. These data are a proxy for the amount of media attention that a firm receives and is used to test to what extent data breaches cause increased media coverage, as well as what effect on outcomes any such change in media coverage may have.

\biblio % Needed for referencing to working when compiling individual subfiles - Do not remove
\end{document}