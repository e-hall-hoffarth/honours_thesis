\documentclass[../Main.tex]{subfiles}
\begin{document}

This paper evaluates a number of mechanisms through which firms may incur losses as a result of suffering a data breach. Using a database of data breaches in the Unites States spanning the years 2005-2017, effects on revenues, expenses, profits and stock returns were tested. There is some evidence that consumers will punish firms for leaking their private data, especially those firms that do so repeatedly, and smaller firms. However, these results are noisy and seem to be driven by some large outliers, because the average effects do not seem to be significant. On the other hand, there is no evidence that firms systematically increase advertising, security, or other expenditures as a result of a data breach. Likewise, there is no evidence that data breaches result in large non-operating expenses such as legal expenses. The fact that instead, operating and particularly advertising expenses may decrease after a data breach seems to be indicative of temporary downsizing or cutbacks. Consistent with the observation that the negative effect on revenues is larger than that on operating expenses, there is evidence that net income falls, but only temporarily in the wake of a data breach. 

In contrast to the lukewarm response of variables in the \textit{firm study}, the \textit{stock market study} demonstrated a strong negative response in firms' stock returns to data breaches. The strength of this response does not seem to be changing over time, nor does it appear to depend on the severity of the data breach. Furthermore, the stock market response seems to be driven by the information about the firm that the data breach reveals, rather than any future losses that the firm may face. Taken together the negative effects of a data breach seem sufficient to justify the fall in stock market value, so this does not seem to represent an arbitrage opportunity.

Considering these conclusions in a broader context, the results of this paper imply that firms do --- to a limited degree --- face corrective action from market forces for leaking data. However, these effects do not seem to be especially strong, and there is no evidence that firms actually respond by taking preventative or remedial action. This does not necessarily mean that there is market failure as forecast by \cite{roberds2009}. It is possible that firms choose not to invest in increased security, advertising, or other measures because such measures are in fact to be ineffectual, or because firms already allocated an efficient amount of spending to these areas before the data breach. Finally, it is possible that firms do indeed increase investment in these measures, but that the increase in spending is proportionally so small that it does not stand out against noise in this study. 

This paper illuminates a number of avenues for future research. Firstly, there is a need for research into factors which predict the risk for data breaches. While this paper attempted to skirt around the issue of endogeneity of data breaches, if it is possible to identify an exogenous source of variation in the risk for data breaches it would allow studies like this one to more strongly identify their results. Secondly, in order to evaluate the possibility of market failure, further research will be required to understand the preventative and remediation strategies available to firms, and their cost effectiveness\footnote{I put this second because conducting this research may first require a strong understanding of the risk factors.}. Finally, in order to provide strong evidence that governments should step in to ensure privacy protections, future studies should attempt to estimate not only the direct cost but also the broader social costs of data breaches including all stakeholders such as firms, consumers, and public institutions, as this is necessary to calculate the socially optimal amount of security investment.

\biblio % Needed for referencing to working when compiling individual subfiles - Do not remove
\end{document}