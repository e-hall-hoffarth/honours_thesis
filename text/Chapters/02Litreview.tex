\documentclass[../Main.tex]{subfiles}
\begin{document}

There are many aspects that require study in order to understand the costs associated with data breaches. Thus far, most of the academic literature in this area has focused on the impact of data breaches on the stock market. For example, \cite{goel2009} conducted an event study on the stock returns for firms that fell victim to data breaches in the period 2004-2008 and found that these firms lost on average 1\% of their market value in the days following the event. \cite{rosati2019} similarly find significant negative effects on stock returns. Whether or not these effects persist is unclear. A study by \cite{morse2011} found significant negative returns, and that these negative returns persist for at least two years after the data breach. However, \cite{acquisti2006} found that these effects do not persist for long; perhaps only a few days. Another study by \cite{gordon2011} investigated how the effects of data breaches on stock prices have changed over time and found broadly\footnote{By comparing pre and post 9/11 periods.} that the effects have been weakening over time. These studies all lend inspiration to the sources of heterogeneity which can be studied when considering how other outcomes respond to data breaches.

However, how might \textit{consumers} be expected to respond to being informed of a data breach, and for what reasons? One aspect to consider is media coverage that is generated in the wake of a data breach. Since these events generate primarily negative coverage, the most obvious conclusion is that sales would be harmed. Indeed, negative publicity can damage reputations, and reputation has been shown to be a strong predictor of sales \citep{livingston2005}. However, a study by \cite{berger2010} found that the effects of negative publicity on sales can be positive for relatively unknown products while being negative for well-known products. This is explained through the competing mechanisms of increased product awareness yet increased negative sentiment caused by negative publicity. Of particular interest is whether --- given information about a data breach --- consumers will try to punish firms for leaking their information. The limited empirical work which has been carried out in this area seems to suggest that this is not the case. \cite{davis2009} found no significant impact on the web traffic of online businesses as a result of data breaches. They provide a number of possible explanations for this including the fact that consumers are somewhat insured against damages from a data breach\footnote{For example, in the case of credit card theft consumers are usually not liable.} and thus do not sustain substantial losses, and a collective action problem which arises as individual consumers do not believe that they can send a sufficiently strong signal to firms by ceasing their purchases, even if they have sustained substantial losses. 

Given that consumer reactions to data breaches are both theoretically ambiguous and empirically small in other studies, the results of this paper to make a significant contribution to existing literature. Furthermore, this paper also considers the internal costs of data breaches, and area in which there is to my knowledge no existing academic literature.

\biblio % Needed for referencing to working when compiling individual subfiles - Do not remove
\end{document}