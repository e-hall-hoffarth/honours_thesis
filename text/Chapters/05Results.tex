\documentclass[../Main.tex]{subfiles}
\begin{document}

This section will begin with discussion of the results of the \textit{firm study}, and will then conclude with discussion of the \textit{stock market study}.

\subsection{Firm Study}

\subsubsection{Revenue}

Table \ref{rev1} shows the primary specifications for the \textit{firm study} with quarterly revenues as the outcome. Specification (1) shows the effect when no time trend is included. This is the average effect if the data breach on revenues during the entire event period, which in this table is one year (robustness to other event periods will be discussed soon). This coefficient is positive, but close to zero and insignificant. Specification (2) allows for a time trend. In this specification we observe that there is a significant immediate drop in revenues after the data breach as evidenced by the negative coefficient on $\beta_1$, but the positive value for $\beta_2$ implies that this loss in revenues recovers over the event period which is consistent with the conclusion of specification (1) that the average effect over the event window is close to zero. These effects are visualised in Figure \ref{revfig}. To put these effects into perspective, the immediate drop in revenues is about 20\% of the average revenue in this sample. Since the unit of revenue is millions USD, this effect is about \$1 billion which is much larger than estimate of total average costs as given by \cite{ponemon2018}. This is likely because the effect is driven by large outliers, for example, firms that go out of business as a result of the data breach. As evidence of this, when the logarithm of revenue\footnote{Which gives a better estimate of the average effect} is used as shown in Table \ref{logrev1} and Figure \ref{logrevfig}, no significant effect is observed. 

In specification (3) controls are added for both the natural logarithm of the number of records leaked in a data breach and the Google searches index. Neither appears to have a strong impact on responses to data breaches, although the coefficient for records leaked is significantly positive at 10\%\footnote{However, I hesitate to conclude therefore that leaking more records results in less loss in revenue.}. Specification (4) includes interactions with dummies for which quartile a firm's revenue falls in in the first period in which it appears in the data. This is a proxy for the size of the firm. The coefficient on these interactions is decreasingly negative in the quartile of the firm's revenue which suggests that smaller firms suffer more punishment in terms of lost sales than larger firms. Finally, specification (5) includes interactions with dummies for the type of data lost in the data breach. These show that lost revenues are larger when "customer data" is lost --- consistent with the consumer punishment mechanism --- and in particular when social security numbers are lost. This seems reasonable because this is generally considered very sensitive information, the loss of which could be very costly for victims.

Table \ref{rev2} shows the basic specification equivalent to specification (2) of Table \ref{rev1} (no controls) for various lengths of event windows. The negative immediate effect on revenues appears to be robust to these various specifications. Note that the slope of the time trend is decreasing in the length of the event period. This is because firms seem to quickly return to their previous level of revenue and stay there, so extending the event window simply causes the regression line to have a more shallow slope.

Within the \textit{firm study} data there are a number of firms which have more than one data breach. Tables \ref{rev3} and \ref{rev4} consider whether a firm having multiple data breaches results in a heterogeneous response. Table \ref{rev3} is the same as Table \ref{rev2} however the sample is restricted to the first data breach for every firm. The effects are not significant, and the point estimates for the immediate effect ($\beta_1$) are much smaller. Table \ref{rev4} shows the main regression for subsets based on the number of data breaches that a firm has in the sample period. While none of the coefficients are significant, the point estimates are increasingly negative in the number of breaches. These two tables provide some suggestive evidence that any consumer punishment effect is increasing in the number of data breaches a firm has. This is an intuitive result: consumers may readily forgive the first data breach as a random case of misfortune, but if it occurs many times consumers will start to believe that the firm is systematically failing to put in place mechanisms to protect their data. 

Finally, Table \ref{rev5} shows results when the event date is offset from the actual public announcement date. In this placebo test the event periods are one year and therefore do not overlap, so (as discussed in the methodology section) the coefficients are expected to not be significant for the offset event dates. However, this does not appear to be the case, so there may be issues with the first identifying assumption. Of particular concern is the significant coefficient for $\beta_1$ two years before the data breach. This suggests that there may be a negative trend before the actual data breach. However, when considering Figure \ref{revfig} there seems to be little trend before the event, so the "no-trend" assumption is not obviously violated. Of less concern is the coefficient for $\beta_1$ for the event date three years after the actual announcement, which is even larger in magnitude than that for the actual event date. This is most likely to reflect the presence of lagged or "knock-on" effects of data breaches such as the example given earlier which was the CEO being fired.

Overall there seems to be some, but not definitive evidence that firm revenues are adversely affected by data breaches. This is the consumer boycott mechanism for market punishment. However, the effect is not robust to all specifications and there is a large degree of heterogeneity in these results based on the size of the firm, and the number of data breaches a firm has.

\subsubsection{Expenses}

Having discussed the interpretation of results from the \textit{firm study} with firm revenues as the outcome, we now turn to the remaining outcomes which are expenses and net income. For these I will not go into depth in interpreting all of the coefficients as the logic is exactly the same as in the previous section. The specifications shown for these outcomes are the same and are shown in the same order in the appendix, and therefore the reader should now be equipped to interpret these. Therefore, I will simply draw attention to some important results without exhaustively discussing every table provided.

Turning first to non-operating income/expenses, Table \ref{nop2} shows surprisingly that for at least short time periods there is a positive coefficient on the immediate effect of the data breach --- implying in contrast to theoretical predictions that the data breach either causes an increase in non-operating income or a decrease in non-operating expenses\footnote{The outcome here is non-operating income net of non-operating expenses, so a positive coefficient actually implies that expenses are falling.}. However, these results are only significant at the 10\% significance level. Even more surprisingly, this positive effect on non-operating income seems to be strongest for firms that have had many data breaches as shown in Table \ref{nop4}. In any case, no specifications imply a strong negative effect, so there seems to be no evidence for the legal expenses mechanism of market punishment. It is still possible that firms on average do indeed face substantial legal costs, however, they are transitory in nature, which makes them difficult to pick up given the temporal resolution available to this study.

For operating expenses, Table \ref{xopr2} shows that for a number of event window lengths there is some negative immediate effect which although insignificant, is consistent across specifications. Figure \ref{xoprfig} shows no trend before the data breach, and clearly shows the discontinuous drop after. These findings support the conclusion that firms may be forced to downsize or otherwise scale back operations somewhat in the aftermath of a data breach --- but only temporarily. In table \ref{xopr5}, the only significant coefficient in this table is for the three years after offset suggesting that any such downsizing either comes as the result of "knock-on" or lagged effects rather than immediate effects of the data breach. In any case, there is certainly no evidence that firms make an effort to improve security or advertise more after a data breach, which would result in increased operating expenses. In fact, specification (1) of Table \ref{otheroutcomes} shows that a proxy for advertising expenditure displays a significant \textit{negative} discontinuous change of about half of its mean after the data breach.

\subsubsection{Profits}

The final outcome to consider for the \textit{firm study} is net income. Negative immediate effects on both revenues and operating expenses have already been discussed, however, the effect on revenue seems to be larger in magnitude and more significant, so it is expected that there will be a net negative effect of data breaches on profits. This is indeed what is observed. Table \ref{pr2} shows that there is a negative discontinuous effect of data breaches on profit which is significant and robust to different event window lengths. The effect is large, at around 90\% of the average quarterly profit in the sample. However, note that this effect is not permanent, both because of the zero effect when no time trend is included, as well as the positive coefficient on the time trend when it is included. We can verify, by comparing Tables \ref{rev2}, \ref{xopr2}, and \ref{pr2} that the immediate effect on net income roughly equals the sum of the immediate effect on revenue and that on operating expenses. Table \ref{pr1} mirrors the results for revenues in many ways: the loss in profits seems to not depend on media coverage as proxied by Google searches or the number of records leaked, but does depend on "consumer data" --- especially social security numbers --- being leaked. However in contrast to the effect on revenue, there is no strong evidence that the effect on profit varies heavily based on the size of the firm. When considering only the first data breach as shown in Table \ref{pr3} the effect is much weaker, and Table \ref{pr4} shows roughly that the effect is increasingly negative in the number of data breaches. This is consistent with the theoretical prediction, as well as the empirical observation for other outcomes that market punishment is stronger for firms that have multiple data breaches.

There are unfortunately some issues with these results. Table \ref{pr5} shows that the placebo test may fail. The coefficients for event date before the data breach have much smaller point estimates than the actual event date, but they are still significant. Furthermore, Figure \ref{prfig} fails to visually depict the same effect that is implied by the regression results --- we should be able to see a lower intercept after the data breach. Nonetheless, this figure also depicts no negative trend before the data breach which will assuage concerns that profits were declining before the data breach, so the "no-trend" assumption is not obviously violated. As with other outcomes, the three years after event date gives a large coefficient, suggesting lagged or "knock-on" effects. 

\subsubsection{Other Outcomes}

In addition to the main outcomes, I also ran the basic regression on a number of additional outcomes available in the COMPUSTAT database. Results are reported in Table \ref{otheroutcomes}. Here there is evidence that shareholders' equity fall in the wake of a data breach. This is consistent with the results of the \textit{stock market study}. Interestingly, $\beta_1$ is negative with Google searches as the outcome. This is likely because any media effects are short lived, and have dissipated by the end of the month (the resolution of this data). Instead, what the regression picks up is a medium term decline in interest the firm consistent with the downsizing hypothesis and reduced advertising expenditure. The fact that data breaches do not appear to cause an increase in Google searches may explain why Google searches do not seem to be an important source of variation in the responses of other variables.

\subsection{Stock Market Study}

Results for the \textit{stock market study} are given in section 8 of the appendix. Figures \ref{capmfig} and \ref{fffig} plot the average CAR for data breaches over a 10 day event period for the CAPM and Fama-French models respectively. While the discontinuity is not completely clean at the event date, it is clear that CAR drops significantly over the event period. This conclusion is supported by the t-test shown in Tables \ref{carttestcapm}, \ref{carttestff}, \ref{carttesttwodaymm} and, \ref{carttesttwodayff}. The former two tables show the effect for the full sample, while the latter two show a sub-sample for which the largest change in CAR over the event period happened within two days of the data breach as discussed in the methodology section. For all specifications with event periods under 10 days CAR is significantly negative at at least the 10\% level with most significant at the 5\% level. So the finding that firms' stock returns fall in the wake of a data breach is very robust and consistent with previous literature on the subject. The Fama-French models provide lower p-values than the CAPM models which suggests that for these data the additional factors succeed in reducing noise. 

When considering the longer event windows (90 and 180 days) the effects become less clear. The specifications using the \textit{two day restriction} do not show significant results for these periods, however, this could well be the result of lack of power from the small sample size that they have access to (51 and 24 respectively). The unrestricted sample for both models shows a significantly negative CAR up to 90 days but not to 180. In particular, the point estimate for the 180 day event window in table \ref{carttestcapm} is very close to zero. Keeping in mind the qualifications about long event windows mentioned in the methodology section, there does not seem to be any evidence in the results of this study that the stock market effects of data breaches persist even six months after.

Finally, Table \ref{smregression} shows the results of regressing the five day CAR from both models on some variables using simple OLS. The zero coefficient in specifications (1) and (4) indicates that there is no trend for the amount of stock market punishment to increase or decrease over the sample period. This contradicts the conclusion of \cite{gordon2011}. The nearly zero coefficients in specifications (2) and (5) indicates that the amount of stock market punishment does not depend strongly on the amount of records leaked\footnote{However note that it is possible that the scale of a data breach may not always be understood five days after the announcement.}. Specifications (3) and (6) regress CAR on the types of data lost. There is only one significant coefficient, which is for credit card data, and this is only significant for one model, so it is most likely spurious. What is interesting to note here is the fact that CAR does not seem to react strongly to consumer data being leaked, even though as it has been already discussed most firm specific variables do depend strongly on this. This observation is evidence for the mechanism whereby stock returns fall because investors internalise information about the firm's infrastructure and management rather than changing their expectations of future earnings. In some sense, this finding is consistent with the results of the \textit{firm study}, which showed that all of the negative effects of data breaches appear to be temporary, and therefore, any lost earnings would have only a nominal impact on net present value of future earnings. 

\biblio % Needed for referencing to working when compiling individual subfiles - Do not remove
\end{document}