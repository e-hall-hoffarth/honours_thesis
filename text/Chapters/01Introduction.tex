\documentclass[../Main.tex]{subfiles}
\begin{document}

On September 28, 2018, Facebook (\citeyear{facebook2018}) announced a data breach in which the information of more than 50 million of its users was exposed to hackers.\footnote{I would like to acknowledge Anders Wettergreen Gundersen for providing the template used to typeset this thesis, as well as Marek Hlavac for creating the Stargazer package for R which was used to create the tables.} In recent years events like this one have become increasingly prominent in the mainstream media as commentators grapple with the societal implications of a so-called, "post-privacy world." Broad philosophical considerations aside, given their scale and cost, data breaches also raise many interesting and important economic questions.

While the frequency of such events is not necessarily increasing, the total annual costs of these events lies in the tens of billions of dollars in the United States alone \citep{edwards2016}. One study by the Ponemon Institute (\citeyear{ponemon2018}) found that the average cost of each individual data breach to the victim firm is approximately \$4 million. Of particular economic concern is the extent to which various stakeholders internalise these costs. The severity of data breaches provides suggestive evidence that firms may systematically under-invest (relative to socially optimal levels) in preventative and remediation strategies due to insufficient corrective action from market forces. Indeed, \cite{roberds2009} develop a theoretical model which predicts exactly this outcome in equilibrium. If this prediction is true, it would have the clear policy implication that market intervention in order to reduce the risk of data breaches would be advisable, not only in order to protect individual privacy, but also to improve economic outcomes. In order to evaluate this claim from the empirical perspective, it is necessary to understand the costs of data breaches that firms face and the extent to which they are internalised, the total social costs, and how effective preventative strategies actually are. Evaluating all of these would be outside of the scope of this paper, which aims only to open this line of academic inquiry by addressing the types and scale of costs of data breaches faced by firms. The remaining issues are left as topics for future research.

In order to address this question, this paper employs an event study methodology which can identify the effect of data breaches on a number of firm specific and stock market variables. I find some evidence that consumers do indeed punish firms for leaking their information, as a discontinuous drop in revenue can be observed after a data breach. However, this effect is short-lived and highly variable. This drop in revenue does indeed translate to an observable temporary loss in net income\footnote{I use profit and net income interchangeably in the context of this paper.}. The amount punishment that a firm faces seems to be increasing in the number of data breaches a firm has. Furthermore, consistent with other studies conducted in this area, I find that stock returns fall precipitously in the wake of a data breach, however, there is no evidence that this effect persists. Despite these costs, there is no evidence that firms increase spending on either security or advertising efforts after a data breach --- providing suggestive evidence that market forces are not enough to cause firms to change their behaviour and achieve the socially optimal outcome. This is perhaps hardly surprising given that all of the negative effects of data breaches appear to be temporary.

The remainder of this paper will proceed as follows: Section 2 provides an overview of relevant past research. Section 3 describes the empirical estimation strategy used, and without giving a full theoretical model, outlines some intuition that guides the expected results. Section 4 discusses the data used. Section 5 analyses the empirical results, all of which are given in the appendix. Finally, Section 6 summarises the conclusions of this study.

\biblio % Needed for referencing to working when compiling individual subfiles - Do not remove
\end{document}