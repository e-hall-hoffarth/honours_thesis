\documentclass[../Main.tex]{subfiles}
\begin{document}

In order to evaluate all of the potential costs of a data breach as outlined in the introduction, this paper considers two separate models applied to two separate data sets. The first model examines both the internal and external costs that a firm may face. This will be referred to as the \textit{firm study} from the remainder of this paper. The second model examines stock market responses to data breaches. This will be referred to as the \textit{stock market study} for now on. Both studies involve an event study methodology, which attempts to identify a discontinuous change in outcomes before and after the data breach. However, the exact implementations differ greatly, so I will discuss them separately. I begin now by outlining the \textit{firm study}.

\subsection{Firm Study}

The \textit{firm study} employs an event study methodology to identify the causal impact of data breaches on affected firms. The model considers both whether there is any immediate effect, and if so, whether those effects persist. This model uses the following regression equation:
\begin{equation}
    \label{firmstudyequation}
    Y_{iq} = \beta_0 + \beta_1 A_{iq} + \beta_2 ( Q \times A_{iq} ) + \vec{\beta_3} X_{iq} + \vec{\beta_4} ( X_{iq} \times \vec{A_{iq}} ) + \alpha_i + \delta_q + \epsilon_{iq}
\end{equation}
Where $Y_{iq}$ is the outcome of interest, $A_{iq}$ is a dummy variable which is 1 if a firm $i$ has already announced a data breach in quarter $q$, $Q$ is the number of quarters since the start of the data, $X_{iq}$ is a matrix of controls, $\vec{A_{iq}}$ is a column vector of $A_{iq}$, and $\alpha_i$ and $\delta_q$ are firm and quarter fixed effects, respectively. In this equation there are two primary coefficients of interest: $\beta_1$ and $\beta_2$. $\beta_1$ measures the immediate discontinuous effect of the data breach on the outcome. Meanwhile, $\beta_2$ is a linear time trend interaction which indicates the persistence of any effect. Mathematically, $\beta_1$ amounts to a change in intercept of the regression function after the data breach, and $\beta_2$ a change in slope. For example, if $\beta_1$ is negative, and $\beta_2$ is positive, this means that there is some immediate negative effect, but that that effect decays over time. If $\beta_1$ is negative, but $\beta_2$ is zero, then the effect persists over the event window. I also include specifications where $\beta_2$ is excluded. Here $\beta_1$ captures the average effect over the event window, so it will be different from zero if there is an effect which persists over the event window, and zero otherwise. Clearly, whether or not a given effect is considered to be persistent depends on the length of the event window considered, since, for example, it is possible for an effect to persist for one year but not two years. Therefore, specifications are given for the primary outcomes with event windows of six months, one year, two years and, three years after the data breach. 

In the \textit{firm study} there are three primary outcomes for which this model is evaluated, which I will now discuss in turn. The first is the revenues of firms. In this case, there are theoretical reasons to believe that $\beta_1$ may be negative or zero. On one hand, if the representative consumer either believes that data breaches occur randomly, and thus the occurrence of a data breach conveys no information about the firm and its chances of being breached again, or they do not suffer disutility from the firm leaking data, then it would be irrational for the consumers to boycott that firm because their expected utility from continuing to buy from that firm has not changed. If this is true then $\beta_1$ and $\beta_2$ are expected to be zero. On the other hand, if the representative consumer internalises information from the data breach such that they believe that the firm is more likely to leak information again in the future, and if leaks cause consumers disutility, then it would be entirely rational for consumers (if they behave as a cohesive unit) to reduce future purchases from that firm as they expect that another costly data breach may occur in the future. Furthermore, consumers (again as a cohesive unit) may wish to boycott as a form of punishment since the credible threat of this punishment if large enough will be enough to discipline all firms into better protecting customer data. If this is true $\beta_1$ is expected to be negative. However, even if it would benefit consumers as a whole to punish firms, they may face a collective action problem, because each individual consumer has an incentive to continue buying from a firm, in the hopes that other consumers will carry out the punishment. Despite this slight theoretical ambiguity, in light of the findings of \cite{davis2009} and the likelihood of a collective action problem it seems more likely \textit{a priori} that $\beta_1$ in the case of firm revenue will not be significantly different from zero.

The other primary outcomes that are considered are costs to and profits of firms as measured by both operating and non-operating expenses, and net income respectively. The interpretation of the coefficients are the same as for revenue, so what remains is to discuss the theoretical effect of data breaches on these outcomes. It seems likely \textit{ex ante} that expenses, particularly non-operating expenses which encompass unexpected and unusual costs such as legal settlements will increase in the wake of a data breach. Other potential sources of increased costs --- which are captured by operating expenses --- include security upgrades, and increased advertising to recover public goodwill. However, these expenses may be transitory in nature --- when considering a longer time-frame expenses may decrease because the data breach causes a general decline in the firm which leads to downsizing. Given lost sales and potentially increased expenses, profits are likely to be adversely impacted. Including this outcome will check the robustness of the revenue and operating expense specifications, as the sum of the effect on these two should roughly equal the effect on profit. 

In addition to considering the magnitude of effects on these primary outcomes, this study also investigates some potential sources of heterogeneity among the responses to data breaches by interacting various outcomes with the $A_{iq}$ dummy. These variables are included in the matrix $X$ of controls, and therefore, their marginal impact on the total effect of a data breach is given by the coefficients in $\vec{\beta_4}$\footnote{Note that for time invariant features such as characteristics of the data breach itself only the interaction term is included because the linear term is co-linear with the firm fixed effect for firms with one data breach.}. In particular, I test whether the size of the data breach (as measured by the number of records leaked), media coverage (as proxied for by Google Trends), the size of the firm, or the type of data leaked result in a heterogeneous response. Finally, the study also considers whether the response in outcomes depends on the number of times a firm leaks data by breaking results out into subsets based on the number of data breaches a firm has had in the sample. I will not now express a prior as to the expected effect of these variables, and instead leave it as a strictly empirical exercise. The observed marginal impact of these factors will be discussed in the results section.

\subsection{Stock Market Study}

The \textit{stock market study} will follow standard procedures for conducting an event study in financial economics. At least two models will be applied in order to test robustness, which are the capital asset pricing (hereafter CAPM) and Fama-French 5 factor models \citep{fama2015}. These models will be fit to the data during the \textit{prediction window}\footnote{I use the period 180 days before the data breach} before the data breach and be used to calculate predicted returns during the \textit{event window} after the data breach, which when subtracted from the actual returns give abnormal returns (AR). These abnormal returns can be summed over the \textit{event window} to generate cumulative abnormal return (CAR). The general logic of this test is that if AR tend to be very large (and thus as is CAR) in absolute value then this means that the event – in this case data breaches – has had a strong effect on returns. There are a number of non-parametric methods to test the significance of these abnormal returns as their distribution can be heavily skewed, however, in order to maintain consistency with the \textit{firm study} and broader common practice in the economics literature, this study will employ a simple t-test. The CAPM has the following functional form:
\begin{equation}
    \label{capmequation}
    R_{it} = \beta_0 + \beta_1 MRP_t + \mu_{it}
\end{equation}
Where $R_{it}$ is the daily excess (greater than risk free) stock return for firm $i$ on day $t$, and $MRP_t$ is value weighted excess market return as given by CRSP. The Fama-French model expands on this with some additional factors to improve fit. It has the following functional form:
\begin{equation}
    \label{ffequation}
    R_{it} = \beta_0 + \beta_1 MRP_t + \beta_2 SMB_t + \beta_3 HML_t + \beta_4 RMW_t + \beta_5 CMA_t + \mu_{it}
\end{equation}
Where SMB, HML, RMW, and CMA are controls for firm size, book-to-market value, profitability and, investment respectively. The coefficients on these terms are not of importance for this study. The theoretical advantage of this model is that it may predict returns over the event period more precisely, potentially allowing for the effect of the data breach to be identified more strongly against noise. For robustness both of these models will be used to calculated AR and subsequently CAR.

The t-test on CAR will determine whether data breaches have a significant effect on stock returns. If the t-statistic is significantly negative then there is evidence of punishment on the stock market in the form of reduced returns as a result of the data breach. Alternatively, if the t-statistic is zero, the firms do not face stock market consequences for the data breach. Furthermore, this study will consider whether these effects persist. If there is a significant effect with a short window, there may or may not be a significant effect with a longer window. Therefore, this study considers event windows of 2, 5 10, 90, and 180 days. Longer event periods are possible, however, over a very long period it becomes difficult to rule out other significant shocks which may occur during the event period.

It is also interesting to consider the possible determinants of any stock market response. This is done by regressing CARs on various features such as number of records leaked via simple OLS. It should be noted however, that these results are merely correlations and do not identify the causal determinants of stock market reactions to data breaches.

This study will compare two mechanisms that can potentially explain stock market responses to data breaches. Firstly, data breaches may cause investors to lower their expectations of future profits as a result of lost sales, and higher expenses --- a change in expectation which is not entirely unwarranted according to the results of the \textit{firm study}. On the other hand, the occurrence of the data breach could signal new and negative information about the quality of a firm's infrastructure and management. This information could cause investors to reassess the fundamental value of the firm. These are both reasons to expect that negative abnormal returns will be observed. 

Given the findings of previous studies \citep{goel2009, morse2011, acquisti2006} it seems likely that some significant negative abnormal stock returns will be observed, at least with short post-event windows. Findings on the persistence of these effects are mixed, and will depend on whether or not data breaches have long-lasting effects, and whether or not the market internalises this information. 

\subsection{Identification}

The regressions in both the \textit{firm model} and the \textit{stock market model} employ an event study methodology and thus have the same identifying assumptions, as well as the same strengths and weaknesses. This approach is in fact quite similar to a difference in difference approach, with the key difference being the absence of an explicit control group --- instead the firms in the sample form their own control for the period before the data breach. Therefore, the identifying assumptions are similar to a difference in difference strategy, with some additional caveats. Firstly, the analogue to the parallel trends assumption relates to the trend for each firm during the \textit{prediction window}. Since each firm is its own control in the \textit{prediction window}, once the firm fixed effects are removed there should be no trend in the residuals before the data breach. Secondly, exactly as with a difference in difference approach, there must be no changes which are simultaneous with the data breach, or any significant exogenous shocks during the event window in question. Thirdly, the information about the data breach must not leak before it is publicly announced, as this is how the event date is defined. Otherwise the coefficients may be biased towards zero because the effect actually started before the regression coefficients can pick it up. 

The first assumption has to do with the validity of the control group. There should be no trend in the outcomes for a firm in the \textit{prediction window}. If this is not the case then there are two distinct issues. Firstly, if there is some trend in an outcome for a firm in the \text{prediction window}, then the coefficients in Equation \ref{firmstudyequation} are biased because this functional form implicitly assumes the slope and intercept of the regression line are zero before the data breach, and then change afterwards. Therefore, the effects of any trend can be wrongfully attributed to the data breach. This is precisely what happens when the parallel trends assumption is not met in a difference in difference approach. Secondly, consider a firm that has declining profits before a data breach. This firm is experiencing structural decline and may have been forced to cut back on spending on key security infrastructure, thus making a data breach more likely. In this case we are concerned that the data breach is endogenously determined and that the effect that is measured is not truly the causal effect of a data breach, but rather the causal effect of a firm being in decline, and thus being more prone to a data breach. 

Given these serious concerns it is fortunate that it is possible to test and verify the first identifying assumption. One way to do so is to simply inspect the plotted residuals for any signs of a trend during the \textit{prediction window}. The other way to do so is to try placebo event dates before the actual data breach and confirm that the coefficients pick up no effect. Both of these tests can be employed and are discussed in the results section. 

On the other hand, the second assumption will be difficult to verify, particularly when considering long event windows. Over a multiple year time period, many omitted factors may change that could introduce bias such as changes of key personnel, changes in competitiveness in industries and so on. In the \textit{firm study} the violation of this assumption could contribute to much of the noise in the observed coefficients. However, this does not necessarily bias estimates in any particular direction, because exogenous shocks could push coefficients in either direction, and the effects of follow up shocks which are endogenous to data breaches (for example the CEO being fired) are --- I would argue fairly --- lumped into the coefficient measuring the effect of the data breach itself. Nevertheless, for the above reasons the time trend coefficients --- especially for long event windows --- will not be as strongly identified as the immediate effect as being the direct causal effect of the data breach.

The third assumption seems plausible, although it is possible that information about a data breach leaks out before the public announcement. This is likely not an issue for the \textit{firm study} because if information about a data breach leaks early it likely does so by a number of days, rather than months, which would not change the fiscal quarter that the true event occurs in. However, this could be significant for the \textit{stock study} which uses daily data. There are two methods used to address this concern. Firstly, a \textit{buffer window} of 5 days between the end of the \textit{prediction window} is included, meaning that the CAR begins to accumulate 5 days before the event date. This means that if the data breach was in fact leaked up to 5 days before it was publicly announced the resulting CAR will still be valid. Secondly, a second specification is offered (Tables \ref{carttesttwodaymm} and \ref{carttesttwodayff}) in which events are excluded if the largest change in CAR did not occur within two days of the event date, suggesting that some other significant event took place\footnote{This is an (admittedly imperfect) way to automate a common practice which is to manually exclude observations where other significant exogenous shocks occur during the event window}. This criteria is harder to satisfy for longer event windows, which is why in this table sample size is decreasing in event window length. The other specification (Tables \ref{carttestcapm} and \ref{carttestff}) does not impose this restriction.

As presented earlier it would be interesting to compare the reaction of consumers and investors to data breach events because the mechanisms through which these parties are affected are vey different. When put on the same scale the investor reaction is expected to be stronger than the consumer reaction, because stock returns price in expectations about consumer behaviour as well as a number of other negative effects of data breaches. Furthermore, if there were no consumer response and yet a stock market response is observed, then this implies that the stock market revaluation is based on factors other than lost sales, such as damage to reputation or lower expectations about infrastructure and management.

\biblio % Needed for referencing to working when compiling individual subfiles - Do not remove
\end{document}

